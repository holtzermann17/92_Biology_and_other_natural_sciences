\documentclass[12pt]{article}
\usepackage{pmmeta}
\pmcanonicalname{CategoryOfMRsystems}
\pmcreated{2013-03-22 18:18:21}
\pmmodified{2013-03-22 18:18:21}
\pmowner{bci1}{20947}
\pmmodifier{bci1}{20947}
\pmtitle{category of $(M,R)$--systems}
\pmrecord{64}{40927}
\pmprivacy{1}
\pmauthor{bci1}{20947}
\pmtype{Topic}
\pmcomment{trigger rebuild}
\pmclassification{msc}{92B99}
\pmclassification{msc}{18-00}
\pmclassification{msc}{93A10}
\pmclassification{msc}{93A30}
\pmclassification{msc}{92B20}
\pmclassification{msc}{92B05}
\pmsynonym{MR-systems}{CategoryOfMRsystems}
\pmsynonym{metabolic-replication systems}{CategoryOfMRsystems}
%\pmkeywords{MR-systems}
%\pmkeywords{categorical algebra of MR-systems}
%\pmkeywords{metabolic-replication in complex biological systems}
%\pmkeywords{molecular biology realizations of GMRs}
%\pmkeywords{nonlinear genetic network models in many-valued $LM_n$ logic algebras}
%\pmkeywords{DNA}
%\pmkeywords{RNAs}
%\pmkeywords{enzymes}
%\pmkeywords{reverse transcriptase}
\pmrelated{MathematicalBiology}
\pmrelated{SystemDefinitions}
\pmrelated{ArtificialInteglligence}
\pmrelated{ComplexSystemsBiology}
\pmrelated{IndexOfCategories}
\pmdefines{MR-system}
\pmdefines{general MR-system}
\pmdefines{MR-quartet}
\pmdefines{morphism of MR-quartets}

% this is the default PlanetMath preamble.  as your knowledge
% of TeX increases, you will probably want to edit this, but
% it should be fine as is for beginners.

% almost certainly you want these
\usepackage{amssymb}
\usepackage{amsmath}
\usepackage{amsfonts}

% used for TeXing text within eps files
%\usepackage{psfrag}
% need this for including graphics (\includegraphics)
%\usepackage{graphicx}
% for neatly defining theorems and propositions
%\usepackage{amsthm}
% making logically defined graphics
%%%\usepackage{xypic}

% there are many more packages, add them here as you need them

% define commands here
\usepackage{amsmath, amssymb, amsfonts, amsthm, amscd, latexsym}
%%\usepackage{xypic}
\usepackage[mathscr]{eucal}

\setlength{\textwidth}{6.5in}
%\setlength{\textwidth}{16cm}
\setlength{\textheight}{9.0in}
%\setlength{\textheight}{24cm}

\hoffset=-.75in     %%ps format
%\hoffset=-1.0in     %%hp format
\voffset=-.4in

\theoremstyle{plain}
\newtheorem{lemma}{Lemma}[section]
\newtheorem{proposition}{Proposition}[section]
\newtheorem{theorem}{Theorem}[section]
\newtheorem{corollary}{Corollary}[section]

\theoremstyle{definition}
\newtheorem{definition}{Definition}[section]
\newtheorem{example}{Example}[section]
%\theoremstyle{remark}
\newtheorem{remark}{Remark}[section]
\newtheorem*{notation}{Notation}
\newtheorem*{claim}{Claim}

\renewcommand{\thefootnote}{\ensuremath{\fnsymbol{footnote%%@
}}}
\numberwithin{equation}{section}

\newcommand{\Ad}{{\rm Ad}}
\newcommand{\Aut}{{\rm Aut}}
\newcommand{\Cl}{{\rm Cl}}
\newcommand{\Co}{{\rm Co}}
\newcommand{\DES}{{\rm DES}}
\newcommand{\Diff}{{\rm Diff}}
\newcommand{\Dom}{{\rm Dom}}
\newcommand{\Hol}{{\rm Hol}}
\newcommand{\Mon}{{\rm Mon}}
\newcommand{\Hom}{{\rm Hom}}
\newcommand{\Ker}{{\rm Ker}}
\newcommand{\Ind}{{\rm Ind}}
\newcommand{\IM}{{\rm Im}}
\newcommand{\Is}{{\rm Is}}
\newcommand{\ID}{{\rm id}}
\newcommand{\GL}{{\rm GL}}
\newcommand{\Iso}{{\rm Iso}}
\newcommand{\Sem}{{\rm Sem}}
\newcommand{\St}{{\rm St}}
\newcommand{\Sym}{{\rm Sym}}
\newcommand{\SU}{{\rm SU}}
\newcommand{\Tor}{{\rm Tor}}
\newcommand{\U}{{\rm U}}

\newcommand{\A}{\mathcal A}
\newcommand{\Ce}{\mathcal C}
\newcommand{\D}{\mathcal D}
\newcommand{\E}{\mathcal E}
\newcommand{\F}{\mathcal F}
\newcommand{\G}{\mathcal G}
\newcommand{\Q}{\mathcal Q}
\newcommand{\R}{\mathcal R}
\newcommand{\cS}{\mathcal S}
\newcommand{\cU}{\mathcal U}
\newcommand{\W}{\mathcal W}

\newcommand{\bA}{\mathbb{A}}
\newcommand{\bB}{\mathbb{B}}
\newcommand{\bC}{\mathbb{C}}
\newcommand{\bD}{\mathbb{D}}
\newcommand{\bE}{\mathbb{E}}
\newcommand{\bF}{\mathbb{F}}
\newcommand{\bG}{\mathbb{G}}
\newcommand{\bK}{\mathbb{K}}
\newcommand{\bM}{\mathbb{M}}
\newcommand{\bN}{\mathbb{N}}
\newcommand{\bO}{\mathbb{O}}
\newcommand{\bP}{\mathbb{P}}
\newcommand{\bR}{\mathbb{R}}
\newcommand{\bV}{\mathbb{V}}
\newcommand{\bZ}{\mathbb{Z}}

\newcommand{\bfE}{\mathbf{E}}
\newcommand{\bfX}{\mathbf{X}}
\newcommand{\bfY}{\mathbf{Y}}
\newcommand{\bfZ}{\mathbf{Z}}

\renewcommand{\O}{\Omega}
\renewcommand{\o}{\omega}
\newcommand{\vp}{\varphi}
\newcommand{\vep}{\varepsilon}

\newcommand{\diag}{{\rm diag}}
\newcommand{\grp}{{\mathbb G}}
\newcommand{\dgrp}{{\mathbb D}}
\newcommand{\desp}{{\mathbb D^{\rm{es}}}}
\newcommand{\Geod}{{\rm Geod}}
\newcommand{\geod}{{\rm geod}}
\newcommand{\hgr}{{\mathbb H}}
\newcommand{\mgr}{{\mathbb M}}
\newcommand{\ob}{{\rm Ob}}
\newcommand{\obg}{{\rm Ob(\mathbb G)}}
\newcommand{\obgp}{{\rm Ob(\mathbb G')}}
\newcommand{\obh}{{\rm Ob(\mathbb H)}}
\newcommand{\Osmooth}{{\Omega^{\infty}(X,*)}}
\newcommand{\ghomotop}{{\rho_2^{\square}}}
\newcommand{\gcalp}{{\mathbb G(\mathcal P)}}

\newcommand{\rf}{{R_{\mathcal F}}}
\newcommand{\glob}{{\rm glob}}
\newcommand{\loc}{{\rm loc}}
\newcommand{\TOP}{{\rm TOP}}

\newcommand{\wti}{\widetilde}
\newcommand{\what}{\widehat}

\renewcommand{\a}{\alpha}
\newcommand{\be}{\beta}
\newcommand{\ga}{\gamma}
\newcommand{\Ga}{\Gamma}
\newcommand{\de}{\delta}
\newcommand{\del}{\partial}
\newcommand{\ka}{\kappa}
\newcommand{\si}{\sigma}
\newcommand{\ta}{\tau}
\newcommand{\med}{\medbreak}
\newcommand{\medn}{\medbreak \noindent}
\newcommand{\bign}{\bigbreak \noindent}
\newcommand{\lra}{{\longrightarrow}}
\newcommand{\ra}{{\rightarrow}}
\newcommand{\rat}{{\rightarrowtail}}
\newcommand{\oset}[1]{\overset {#1}{\ra}}
\newcommand{\osetl}[1]{\overset {#1}{\lra}}
\newcommand{\hr}{{\hookrightarrow}}
\begin{document}
\subsection{Metabolic-Replication Systems}

 \PMlinkexternal{Robert Rosen}{http://planetphysics.org/encyclopedia/RobertRosen.html} introduced \emph{metabolic--repair models}, or $(M,R)$-systems in mathematical biology 
\PMlinkexternal{(\emph{abstract relational biology})}{http://planetphysics.org/encyclopedia/AbstractRelationalBiologyARB.html} in 1957 (\cite{RRosen1, RRosen2}); such systems will be here abbreviated as $MR$-systems, (or simply $MR$'s). Rosen, then represented the $MR$'s in terms of categories of sets, deliberately selected without any structure other than the {\em discrete topology of sets}.

Theoreticians of life's origins postulate that Life on Earth has begun with the simplest possible organism, called the \emph{primordial}. Mathematicians interested in biology and this important question of the minimal living organism have
attempted to define the functional relations that would have made life possible in a such a minimal system-- a grandad
and granma of all living organisms on Earth.   

\begin{definition}
The simplest $MR$-system is a relational model of the primordial organism which is defined by the following \emph{categorical sequence (or diagram) of sets and set-theoretical mappings}:
$f: A \rightarrow B, \phi: B \rightarrow Hom_{MR}(A,B)$, where $A$ is the set of inputs to the 
$MR$-system, $B$ is the set of its outputs, and $\phi$ is the `repair map', or $R$-component, of the $MR$-system which associates to a certain product, or output $b$, the `metabolic' component (such as an enzyme, E, for example)
represented by the set-theoretical mapping $f$. Then, $Hom_{MR}(A,B)$ is defined as the set of all such metabolic (set-theoretical) mappings (occasionally written incorrectly by some authors as $\left\{f\right\}$). 
\end{definition}

\begin{definition} 
A \emph{general $(M,R)$-system} was defined by Rosen (1958a,b) as the network or graph of the metabolic and repair components that were specified above in \textbf{Definition 0.1}; such components are networked in a complex, abstract `organism' defined by all the abstract relations and connecting maps between the sets specifying all the metabolic and repair components of such a general, abstract model of the biological organism. The mappings bettwen 
$(M,R)$-systems are defined as the the metabolic and repair set-theoretical mappings, such as $f$ and $\phi$ (specified in \textbf{Definition 0.1}); moreover, there is also a finite number of sets (just like those that are defined as in \textbf{Definition 0.1}): $A_i, B_i$, whereas $f \in Hom_{MR_i}(A_i,B_i)$ and 
$\phi \in Hom_{MR_i}[B, Hom_{MR_i}(A_i,B_i)]$,  with $i \in I$, and $I$ being a finite index set, or directed set, with $(f,\phi)$ being a finite number of distinct metabolic and repair components pairs. Alternatively, one may think of a a general $MR$-system as being `made of' a finite number $N$ of interconnected $MR_i$, metabolic-repair modules with input sets $A_i$ and output sets $B_i$. To sum up: 
a \emph{general MR-system} can be defined as a \emph{family of interconnected quartets}:
$\left\{(A_i, B_i, f_i, \phi_i)\right\}_{i \in I}$, where $I$ is an index set of integers $i=1, 2, ..., n$.
\end{definition}

\subsection{Category of (M,R)--systems}

\begin{definition}
 
 A \emph{category of $(M,R)$-system quartet modules}, $\left\{(A_i, B_i, f_i, \phi_i)\right\}_{i \in I}$, with I being an index set of integers $i=1,2,..., n$, is a small category of sets with set-theoretical mappings defined by the MR-morphisms between the quarted modules $\left\{(A_i, B_i, f_i, \phi_i)\right\}_{i \in I}$, and also with repair components defined as $\phi_i \in Hom_{MR_i}[B, Hom_{MR_i}(A_i,B_i)]$, with the $(M,R)$-morphism composition defined by the usual composition of functions between sets.  

 With a few, additional notational changes it can be shown that the category of $(M,R)$-systems  
is a subcategory of the category of automata (or sequential machines), $\mathcal{S}_{[M,A]}$ (\cite{ICB73, ICBM74}).
\end{definition}

\subsubsection{Remarks:}
For over two decades, Robert Rosen developed with several coworkers the MR-systems theory and its applications
to life sciences, medicine and general systems theory.  He also considered biocomplexity to be an `emergent', defining feature of organisms which is \emph{not reducible in terms of the molecular structures} (or molecular components) of the organism and their physicochemical interactions. However, in his last written book in 1997 on \emph{``Essays on Life Itself", published posthumously in 2000}, Robert Rosen finally accepted the need for representing organisms in terms of {\em categories with structure} that entail biological functions, both metabolic and repair ones. Note also that, unlike Rashevsky in his theory of organismic sets, Rosen did not attempt to extend the $MR$s to modeling societies, even though with appropriate modifications of \emph{generalized $(M,R)$-system categories with structure} (\cite{ICB73, ICBM74, ICB87a}), this is feasible and yields meaningful mathematical and sociological results. 
Thus, subsequent publications have generalized MR-system (GMRs) and have studied the fundamental, mathematical properties of algebraic categories of GMRs that were constructed functorially based on the {Yoneda-Grothendieck Lemma}
and construction. Then it was shown that such algebraic categories of GMRs are \emph{Cartesian closed} \cite{ICB73}. 
Several \emph{molecular biology realizations of GMRs} in terms of DNA, RNAs, enzymes, 
$RNA \to DNA$-reverse trancriptases, and other biomolecular components were subsequently introduced and discussed in ref. \cite{BBGGk6,ICB87a, ICB87b} in terms of \emph{non-linear genetic network models} in many-valued, $LM_n$ logic algebras (or \PMlinkname{algebraic category $\mathcal{LM}$ of $LM_n$ logic algebras}{AlgebraicCategoryOfLMnLogicAlgebras}).

If simple $(M,R)$-systems are considered as sequential machines or automata the category of $(M,R)$-systems and  $(M,R)$-system homomorphisms is a subcategory of the automata category. However, when $(M,R)$-systems are considered together with their dynamic representations the category of dynamic $(M,R)$-systems is no longer a subcategory of the category of automata.


\begin{thebibliography}{9}

\bibitem{Rashevsky1-yr1965}
Rashevsky, N.: 1965, The Representation of Organisms in Terms of
Predicates, \emph{Bulletin of Mathematical Biophysics} \textbf{27}: 477-491.

\bibitem{Rashevsky2-1969}
Rashevsky, N.: 1969, Outline of a Unified Approach to Physics, Biology and Sociology., \emph{Bulletin of Mathematical Biophysics} \textbf{31}: 159--198.

\bibitem{Rosenbook}
Rosen, R.: 1985, \emph{Anticipatory Systems}, Pergamon Press: New York.

\bibitem{RRosen1}
Rosen, R.: 1958a, A Relational Theory of Biological Systems \emph{Bulletin of Mathematical Biophysics} 
\textbf{20}: 245-260.

\bibitem{RRosen2}
Rosen, R.: 1958b, The Representation of Biological Systems from the Standpoint of the 
Theory of Categories., \emph{ Bulletin of Mathematical Biophysics} \textbf{20}: 317-341.

\bibitem{RRosen3}
Rosen, R.: 1987, On Complex Systems, \emph{European Journal of Operational Research} \textbf{30}:129--134.

\bibitem{ICB73}
Baianu, I.C.: 1973, Some Algebraic Properties of \emph{\textbf{(M,R)}} -- Systems. \emph{Bulletin of Mathematical Biophysics} \textbf{35}, 213-217.

\bibitem{ICBM74}
Baianu, I.C. and M. Marinescu: 1974, On A Functorial Construction of \emph{\textbf{(M,R)}}-- Systems. \emph{Revue Roumaine de Mathematiques Pures et Appliqu\'ees} \textbf{19}: 388-391.

\bibitem{ICB80}
Baianu, I.C.: 1980, Natural Transformations of Organismic Structures.,
\emph{Bulletin of Mathematical Biology},\textbf{42}: 431-446.

\bibitem{ICB77}
I.C. Baianu: 1977, A Logical Model of Genetic Activities in \L{}ukasiewicz Algebras: The Non-linear Theory. \emph{Bulletin of Mathematical Biophysics}, \textbf{39}: 249-258.

\bibitem{ICB8}
I.C. Baianu: 1983, Natural Transformation Models in Molecular Biology., in \emph{Proceedings of the SIAM Natl. Meet}., Denver, CO.; \PMlinkexternal{An Eprint is here available}{http://cogprints.org/3675/1/Naturaltransfmolbionu6.pdf} .

\bibitem{ICB9}
I.C. Baianu: 1984, A Molecular-Set-Variable Model of Structural and Regulatory Activities in Metabolic and Genetic Networks., \emph{FASEB Proceedings} \textbf{43}, 917.

\bibitem{ICB87a}
I.C. Baianu: 1987a, Computer Models and Automata Theory in Biology and Medicine., in M. Witten (ed.), 
\emph{Mathematical Models in Medicine}, vol. 7., Pergamon Press, New York, 1513--1577; \PMlinkexternal{CERN Preprint No. EXT-2004-072:}{http://documents.cern.ch/cgi-bin/setlink?base=preprint&categ=ext&id=ext-2004-067}.

\bibitem{ICB87b}
I.C. Baianu: 1987b, Molecular Models of Genetic and Organismic Structures, in \emph{Proceed. Relational Biology Symp.} Argentina; \PMlinkexternal{CERN Preprint No.EXT-2004-067:MolecularModelsICB3.doc}{http://documents.cern.ch/cgi-bin/setlink?base=preprint&categ=ext&id=ext-2004-067}.

\bibitem{Bgg2}
I.C. Baianu, Glazebrook, J. F. and G. Georgescu: 2004, Categories of Quantum Automata and 
N-Valued \L ukasiewicz Algebras in Relation to Dynamic Bionetworks, \textbf{(M,R)}--Systems and
Their Higher Dimensional Algebra, 
\PMlinkexternal{Abstract of Report is here available as a PDF}{http://www.ag.uiuc.edu/fs401/QAuto.pdf} and 
\PMlinkexternal{html document}{http://doc.cern.ch/archive/electronic/other/ext/ext-2004-058/QuantumAutnu3_ICB.pdf}

\bibitem{BGB2}
R. Brown, J. F. Glazebrook and I. C. Baianu: A categorical and higher dimensional algebra framework for complex systems and spacetime structures, \emph{Axiomathes} \textbf{17}:409--493.
(2007).

\bibitem{LO68}
L. L$\ddot{o}$fgren: 1968. On Axiomatic Explanation of Complete Self--Reproduction. \emph{Bull. Math. Biophysics}, 
\textbf{30}: 317--348. 

\bibitem{ICB2004a}
Baianu, I.C.: 2004a. \L{}ukasiewicz-Topos Models of Neural Networks, Cell Genome and Interactome Nonlinear Dynamic Models (2004). Eprint. Cogprints--Sussex Univ. 

\bibitem{ICB04b}
Baianu, I.C.: 2004b \L{}ukasiewicz-Topos Models of Neural Networks, Cell Genome and Interactome Nonlinear Dynamics). CERN Preprint EXT-2004-059. \textit{Health Physics and Radiation Effects} (June 29, 2004). 
 
\bibitem{ICB2k6}
Baianu, I. C.: 2006, Robert Rosen's Work and Complex Systems Biology, \emph{Axiomathes} \textbf{16}(1--2):25--34.

\bibitem{BBGGk6}
Baianu I. C., Brown R., Georgescu G. and J. F. Glazebrook: 2006, Complex Nonlinear Biodynamics in Categories, Higher Dimensional Algebra and \L{}ukasiewicz--Moisil Topos: Transformations of Neuronal, Genetic and Neoplastic Networks., \emph{Axiomathes}, \textbf{16} Nos. 1--2: 65--122.

\end{thebibliography}
%%%%%
%%%%%
\end{document}
