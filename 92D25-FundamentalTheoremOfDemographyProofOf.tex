\documentclass[12pt]{article}
\usepackage{pmmeta}
\pmcanonicalname{FundamentalTheoremOfDemographyProofOf}
\pmcreated{2013-03-22 13:24:42}
\pmmodified{2013-03-22 13:24:42}
\pmowner{aplant}{12431}
\pmmodifier{aplant}{12431}
\pmtitle{fundamental theorem of demography, proof of}
\pmrecord{10}{33957}
\pmprivacy{1}
\pmauthor{aplant}{12431}
\pmtype{Proof}
\pmcomment{trigger rebuild}
\pmclassification{msc}{92D25}
\pmclassification{msc}{37A30}

\usepackage{amssymb}
\usepackage{amsmath}
\usepackage{amsfonts}

\def\Point{$\bullet$ }
\def\ie{\emph{i.e.},}
\begin{document}
\Point First we will prove that there exist $m, M>0$ such that
\begin{equation}
m\leq \frac{\|x_{k+1}\|}{\|x_k\|}\leq M
\label{ineq:Golub0}
\end{equation}
for all $k$,
with $m$ and $M$ \PMlinkescapetext{independent} of the sequence. In \PMlinkescapetext{order} to show this we 
use the primitivity of the matrices $A_k$ and $A_\infty$.
Primitivity of $A_\infty$ implies that there exists
$l\in\mathbb{N}$ such that
\[
A^l_\infty\gg 0
\]
By continuity, this implies that there exists $k_0$ such that, for all 
$k\geq k_0$, we have
\[
A_{k+l}A_{k+l-1}\cdots A_k\gg 0
\]
Let us then write $x_{k+l+1}$ as a function of $x_k$:
\[
x_{k+l+1}=A_{k+l}\cdots A_kx_k
\]
We thus have
\begin{equation}
\|x_{k+l+1}\|\leq C^{l+1} \|x_k\|
\label{ineq:Golub1}
\end{equation}
But since the matrices $A_{k+l}$,\ldots,$A_k$ are strictly positive
for $k\geq k_0$, there exists a $\varepsilon>0$ such that each
\PMlinkescapetext{component} of these matrices is superior or equal to 
$\varepsilon$. From this we deduce that
\[
\|x_{k+l+1}\|\geq \varepsilon \|x_k\|
\]
for all $k\geq k_0$. 
Applying \PMlinkescapetext{relation} (\ref{ineq:Golub1}), we then have that
\[
C^{l} \|x_{k+1}\|\geq \varepsilon \|x_k\|
\]
which yields
\[
\|x_{k+1}\|\geq \frac{\varepsilon}{C^l}\|x_k\|
\]
for all $k\geq 0$,
and so we indeed have \PMlinkescapetext{relation} (\ref{ineq:Golub0}).

%%%%
\Point Let us denote by $e_k$ the (normalised) Perron eigenvector of
$A_k$. Thus
\[
A_ke_k=\lambda_ke_k\quad \|e_k\|=1
\]
Let us denote by $\pi_k$ the projection on the supplementary space of
$\{e_k\}$ invariant by $A_k$. Choosing a proper norm, we can find
$\varepsilon>0$ such that
\[
\left| A_k\pi_k \right| \leq (\lambda_k-\varepsilon)
\]
for all $k$.
%%%%
\Point We shall now prove that
\[
\frac{\left<e_{k+1}^*,x_{k+1}\right>}{\left<e_k^*,x_k\right>}\to \lambda_\infty \textrm{
  when } k\to\infty
\]
In order to do this, we compute the inner product of the sequence 
$x_{k+1}=A_kx_k$ with the $e_k$'s:
\begin{eqnarray*}
\left<e_{k+1}^*,x_{k+1}\right>&=& \left<e_{k+1}^*-e_k^*,A_kx_k\right>+\lambda_k \left<e_k^*,x_k\right>
\\
&=& o\left(\left<e_k^*,x_k\right> \right)+ \lambda_k \left<e_k^*,x_k\right>
\end{eqnarray*}
Therefore we have
\[
\frac{\left<e_{k+1}^*,x_{k+1}\right>}{\left<e_k^*,x_k\right>}=o(1)+\lambda_k
\]

\Point Now assume
\[
u_k=\frac{\pi_k x_k}{\left<e_k^*,x_k\right>}
\]
We will verify that $u_k\to 0$ when $k\to\infty$. We have
\begin{eqnarray*}
u_{k+1}&=& (\pi_{k+1}-\pi_k)A_k\frac{x_k}{\left<e_{k+1}^*,x_{k+1}\right>} +
\frac{\left<e_k^*,x_k\right>}{\left<e_k^*,x_{k+1}\right>} A_k\pi_k\frac{x_k}{\left<e_k^*,x_k\right>}
\end{eqnarray*}
and so
\[
|u_{k+1}|\leq
|\pi_{k+1}-\pi_k|C'+\frac{\left<e_k^*,x_k\right>}{\left<e_{k+1}^*,x_{k+1}\right>}
(\lambda_k-\varepsilon) |u_k|
\]
We deduce that there exists $k_1\geq k_0$ such that, for all $k\geq k_1$
\[
|u_{k+1}| \leq \delta_k+(\lambda_\infty-\frac{\varepsilon}{2}) |u_k|
\]
where we have noted
\[
\delta_k=(\pi_{k+1}-\pi_k)C'
\]
We have $\delta_k\to 0$ when $t\to\infty$, we thus finally deduce that 
\[
|u_k|\to 0\textrm{ when }k\to\infty
\]
Remark that this also implies that
\[
z_k=\frac{\pi_kx_k}{\|x_k\|}\to 0\textrm{ when } k\to\infty
\]

\Point We have $z_k\to 0$ when $k\to\infty$, and $x_k/\|x_k\|$
can be written
\[
\frac{x_k}{\|x_k\|}=\alpha_ke_k+z_k
\]
Therefore, we have $\alpha_ke_k\to 1$ when $k\to\infty$, which implies 
that $\alpha_k$ tends to 1, since we have chosen $e_k$ to be
normalised (\ie $\|e_k\|=1$).

We then can conclude that
\[
\frac{x_k}{\|x_k\|}\to e_\infty \textrm{ when }k\to\infty
\]
and the proof is done.
%%%%%
%%%%%
\end{document}
