\documentclass[12pt]{article}
\usepackage{pmmeta}
\pmcanonicalname{TheHistoryToGettingSettledToCompleteReviews}
\pmcreated{2013-11-27 10:59:44}
\pmmodified{2013-11-27 10:59:44}
\pmowner{jacou}{1000048}
\pmmodifier{}{0}
\pmtitle{The History to Getting Settled To Complete Reviews}
\pmrecord{20}{40622}
\pmprivacy{1}
\pmauthor{jacou}{0}
\pmtype{Definition}
\pmcomment{trigger rebuild}
\pmclassification{msc}{92D25}
\pmrelated{MarkovChain}
\pmrelated{Matrix}
\pmrelated{TransitionMatrix}


\begin{document}
The \emph{Leslie model} is an approach to predicting the \PMlinkescapetext{size} of an animal population after $t$ (discrete) \PMlinkescapetext{units} of time. The population is split up into a partition of age \PMlinkescapetext{groups} according to \PMlinkescapetext{differences} in fecundity and survival rates in each \PMlinkescapetext{group}. The population \PMlinkescapetext{growth} data are then related with a \emph{Leslie matrix}. The \PMlinkescapetext{basic} form of a Leslie matrix is as follows:

\theoremstyle{definition}
\newtheorem*{defn}{Definition}
\begin{defn}
Given $n$ age \PMlinkescapetext{categories} for a population and $0 \leq j \leq n$, there is a fecundity rate $F_j$, which is the average number of offspring from a member of \PMlinkescapetext{category} $j$ who live long enough to enter the youngest age \PMlinkescapetext{category} (zero of course) in a single unit of time, and a survival rate $S_j$, which is the percentage of members in the \PMlinkescapetext{category} $j$ who live to enter the \PMlinkescapetext{category} $j+1$ in a single unit of time. These data are entered into a Leslie matrix like so:

\begin{displaymath}
\begin{bmatrix}
  F_1 & F_2 & F_3 & F_4 \\
  S_1 &  0  &  0  &  0  \\
   0  & S_2 &  0  &  0  \\
   0  &  0  & S_3 &  0  \\
\end{bmatrix}
\end{displaymath}

In other \PMlinkescapetext{words}, if $A$ is a Leslie matrix, then $a_{0j} = F_j$ for all $0 \leq j \leq n$ and $a_{(j+1)(j)} = S_j$ for all $0 \leq j \leq n-1$.
\end{defn}

Given an initial population vector $v$ that gives the number of members in each \PMlinkescapetext{category}, the Leslie model predicts that the number of members in each \PMlinkescapetext{category} after $t$ \PMlinkescapetext{units} of time is $(A^{t})v$. The \PMlinkescapetext{unit} of time is customarily (but not necessarily) years.

Note that the Leslie \PMlinkescapetext{model} can be thought of as \PMlinkescapetext{similar} to a Markov chain. The most important \PMlinkescapetext{difference} is that, since reproduction introduces new members into the population, the fecundity and survival rates in any given \PMlinkescapetext{group} do not necessarily add up to one. Also, unlike \emph{most} Markov chains, the next state for any member of the population is of course deterministic. \footnote{A small \PMlinkescapetext{difference} in convention is that the Leslie matrix is usually to the \emph{left} of the initial population vector when the two are multiplied, as compared to a Markov chain where the \emph{initial distribution} is usually to the left. This plays a role in the \PMlinkescapetext{structure} of the matrices in either case.}

\begin{thebibliography}{9}
\bibitem{source}
\PMlinkexternal{Notes for \emph{WLF 448: Fish \& Wildlife Population Ecology}}{http://www.cnr.uidaho.edu/wlf448/Leslie1.htm}
\end{thebibliography}
%%%%%
%%%%%
\end{document}
